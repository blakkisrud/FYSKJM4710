\section{Charged Particles}

\subsection{Kinematics}

An incoming electron (or another charged particle) interacts with an atom (or another charged particle). 
The collision (or interaction) can either result in an excitation or an ionization. 
In the excitation, the atom is elevated to a higher energy state, where as in the ionization case, a fast electron is ejected.
A cross section can be derived using elastic collision formulae, dependent on energy of the incoming particle.
The cross section rises quickly to some energy $E_{peak}$ and decreases gradually to some $E_{max}$
The "stopping power" $S$ is the most crucial parameter.
It describes how the particle looses energy over some distance x.
If kinetic energy is $T$, distance traveled is $x$ and the stopping power is $S$, we can simply write

\begin{equation}
    S = \frac{\mathrm{d}T}{\rho \mathrm{d}x}
\end{equation}

where $\rho$ is the density of the material.

\subsection{Hard and soft collisions}

We separate between soft and hard collisions. 
The soft collisions are collision where the impact parameter is much longer than the atomic radius. 
There are only weak Coulomb-interaction and low amounts of energy are transferred.
Predominantly excitation, but there could be some ionization.
The energy transfer range on the spectrum from $E_{\mathrm{min}}$ to some $H$.

For hard collisions, the particle travels through the atom.
The transfers of energy are large (but few) and the spectral energies are from $H$ and up to $E_{\mathrm{max}}$.
If the binding energy is negligible, it can be considered as an elastic collision between free particles.
If we add them up, we get a "final" stopping power. 
The stopping power increases with the charge of the incoming particle and decreases with the square of the velocity.
There exist some cross-sections that describe hard electron-electron interaction, but they are complicated.
Shell corrections are the corrections when one excludes the simplification that the velocity of the incoming particles are much higher than the atomic electrons.
The K-shell electrons are most energetic, they are the first that has to be corrected for, then the L-shell and so on.
When this is corrected for, a slightly lower stopping power can be found.
A second correction are done for so called "density effects" in which the particle polarizes the surrounding medium, resulting in weaker interactions.

Linear energy transfer denotes the part of the energy transferred in the surroundings of the particle.
The energy spectrum is integrated up to a certain cut-off $\delta$

\subsection{Radiation loss}

Bremsestrahlung is emitted from a charged particle that traverses the field from the electrons and nuclei. 
The effect from the radiation is given by Larmors formula (from classical electromagnetism) for an accelerated charged particle.
As it depends on the mass ratio squared, it is not important for any other particles than the electron.
Losses from the bremsstrahlung are called radioactive losses. 
The maximal energy lost is up to the total kinetic energy of the particle.
One can define a radiation stopping power, dependent of the square of the $Z$-number (higher $Z$-material leads to higher amount of bremsstrahlung)
The yield from bremsstrahlung is called radiation yield.
Some energy can also be lost from Cerenkov radiation, if the speed of the electrons are higher than the phase speed of light in that medium.
One can also have nuclear interactions where a charged particle is scattered.
An annihilation of positrons (or electron encountered by a positron) can also happen.

\subsection{Range}

The range $\mathcal{R}$ of a charged particle is the expected path-lenght travelled by that particle. The projected range is then the largest depth a particle can travel \emph{into} a medium. 
For electrons the projected range are smaller than the range, and for heavy particles they are roughly equal.
The range can be approximated by using the stopping power and the "continious slowing down approximation"

\begin{equation}
    \mathcal{R}_{\mathrm{CSDA}} = \int \limits_0^{T_0} \left ( \frac{\mathrm{d}T}{\mathrm{d}x} \right )^{-1} \mathrm{d}T
\end{equation}

The range depends on

\begin{enumerate}
    \item 
    The charge and the kinetic energy
    \item
    The density, mean electron density and excitation potential of the absorbing media.
\end{enumerate}

For low $Z$-media, the range and the projected depth will be approximately equal, but at higher and higher $Z$-media, the range will be much longer than the depth.

\subsection{Scattering}

In a beam of charged particles, one can have variations in energy depositions from straggling, and some variations in angular scattering. 
This leads to the beam being "smeared out" when traversing a medium.
The energy will shift downwards and the energy width will be substantially broader.
