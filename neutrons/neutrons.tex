\section{Neutrons}

The neutron, having no charge will behave somewhat like photons.
Neutrons will collide with nuclear targets until they reach thermal equilibrium. 
They are unstable and will disintegrate into a proton, anti neutrino and electron after 12 min.
The two main interactions are

\begin{enumerate}
    \item 
    Scattering. 
    The incoming neutron collides with a nucleus, they both change direction and speed
    \item
    Absorption - the nucleus absorbs the neutron and later deexcitates. This can yield fission products, protons, $\alpha$-particles...
\end{enumerate}

The cross sections of these interactions will depend on the energy of the neutron and the atomic structure.
For low energy neutrons, capturing is most probable. 
Then as the energy of the neutrons increases, elastic scattering becomes more and more probable.
When the neutron energy is high, inelastic scattering occurs, and for even higher energy, more than two particles are ejected. 
At the highest energies, fission becomes possible. 
It can be shown that hydrogen-rich absorbers are the best moderators.
For low energies ($T_n < $ 500 keV), "Potential" and "Resonance"-scattering is possible

\begin{enumerate}
    \item 
    Scattering on the surface of the nucleus.
    \item
    The neutron is absorbed, but reemitted
\end{enumerate}

U$^{235}$ has a high cross-section for thermal neutrons. 
Thermal neutrons is in thermal equlibrium with the surroundings (around 0.025 keV at room temperature).

For high energy neutrons one has inelastic collisions. 
They appear at given energies, giving rise to "resonance" phenomena.
This can lead to the emission of more than one particle, the cross-sections are complicated.
The neutron beam is attenuated exponentially like the photon.

In the light of dosimetry, one can estimate a KERMA-factor and use that to calculate the absorbed dose for neutrons.
Mixed field dosimetry - for later
