\section{Dosimetry concepts and ionometry}

\subsection{Dosimetry methods}

One large group of dosimeters is calorimetry. 
The measurement of an increase of temperature.
This increase is minuscule, as 1 Gy will make a temperature increase in Al of 1 mK.
A circuit is connected to a thermistor, a semiconductor that changes resistance with temperature.
The accuracy has to be high, about 10 $\mathrm{\mu}$K
A general formula for temperature increase could for example be 

\begin{equation}
    \delta T = \frac{E(1-\delta}{h}
\end{equation}

Here, $h$ is thermal capacity.

Pros:

\begin{enumerate}
    \item
        Absolute, the measurement is direct
    \item
        The sensitive volume can be in a wide range of materials
    \item
        It is independent of dose-rate
\end{enumerate}

Cons:

\begin{enumerate}
    \item
        The temperature increase is minute
    \item
        The apparatus tends to be bulky
\end{enumerate}

Another class, closely connected class of dosimeters are ones with thermally activated luminescence. 
These operates by containing crystals that gives off visible light when heated.
One need a way to detect them, possibly a photo-multiplicator tube or similar contraptions. 
The crystals glow with a given wavelength, and the intensity is proportional to the temperature increase. 

They consist of "traps" and "holes" that can hold the electrons for a certain amount of time. 
The traps holds the electrons away from the "hole"
When the electrons are allowed to recombine in so-called "luminescence-centers", light is emitted. 

Pros:

\begin{enumerate}
    \item 
    They are very sensitive
    \item
    Small, reusable
    \item
    Provides rapid readout
    \item
    Avaliable in different materials
    
\end{enumerate}

Cons:

\begin{enumerate}
    \item
    Uniformity is an issue
    \item
    The response is supralinear
    \item
    Fading, light sensitivity
    \item
    Sensitivity changes with exposure
\end{enumerate}


