\section{Cavity theory}

Problem when measuring the dose to a media with a probe.
If the probe and the media have different compositions, the dose will \emph{not} be equal.
Now, start with a $\gamma$-beam.
We have N photons hitting a thin foil, and we have CPE-conditions.
Further more, the attenuation of the photon-beam is close zero.
We have some photons entering, with some electrons, but we assume that for each electron entering, we have another electron exiting.
The transferred energy to the foil is then

\begin{equation}
    \epsilon_{tr} = R_{in,\gamma} + R_{in, e} - R_{out, \gamma} - R_{out, e}
\end{equation}

and under CPE:

\begin{equation}
    \epsilon_{tr} = R_{in, \gamma} - R_{out, \gamma} = N(hv) \mu_{tr} \Delta x
\end{equation}

The absorbed dose is simply the Kerma, 

\begin{equation}
    D = K = \frac{\epsilon_{tr}}{m} = \frac{N(hv) \mu_{tr} \Delta x}{m} = \Upsilon \frac{\mu_{tr}}{\rho}
\end{equation}

And if we allow bremsstrahlung

\begin{equation}
    D = \Upsilon \frac{\mu_{en}}{\rho}
\end{equation}
